\section{系统实现}

\subsection{基础功能实现}

\subsubsection{用户认证}

用户认证包含登录界面,注册界面,用户资料界面。

用户进入系统将自动进入欢迎界面,在该界面中,用户可以选择登录,注册或以游客身份登录。如果以游客身份登录则直接进入系统,使用一个前端内置的账户,仅提供体验功能,无法与后端进行交互。

登录界面提供基础的登录服务,可通过密码和账户名或邮箱名登录。用户勾选``记住我''选项后可将登录记录保留在本地,在有效时间内均可自动进入系统。如果用户直接进入系统,系统会调用浏览器 localhost 数据判断用户近期是否有登录记录,如果有且满足登录条件则自动登录,否则使用游客账户进入系统,代码如下:

\begin{JavaScript}
const getInitUserId = (): string => {
  // 判断上次登录时间是否在一周内
  const checkLoginData = (loginData: Date): boolean => {
    const now = new Date();
    const spanDay = getTimeSpan(now, loginData);
    return spanDay <= 7;
  };
  const userId = localStorage.getItem("userId");
  const lastLoginDate = localStorage.getItem("lastLoginDate");
  // 存在用户记录且满足自动登录条件
  if (
    userId !== null &&
    lastLoginDate !== null &&
    checkLoginData(new Date(lastLoginDate))
  )
    return userId;
  // 不存在,则用游客身份登录
  else return "visitor";
};
\end{JavaScript}

此外界面也可跳转到注册界面或直接以游客身份进入系统,登录界面如图\ref{fig:登录界面}所示:

\begin{figure}[H]
  \small
  \centering
  \begin{tikzpicture}[font=\footnotesize]
    \begin{scope}[xshift=0cm]
      \node () at (-7,0) {};
      \node () at (7,0) {};
      \node [draw=black!60] (fig) at (0,0) {\includegraphics[width=4cm]{./figs/login.jpg}};
      \draw [Circle-] (-1.75,-0.3) -- (-3,-0.3) node [left] {输入用户名或邮箱};
      \draw [Circle-] (-1.75,-0.8) -- (-3,-0.8) node [left] {输入密码};
      \draw [Circle-] (-1.75,-1.5) -- (-3,-1.5) node [left] {点击提交表单};
      \draw [Circle-] (-1.75,-2.1) -- (-3,-2.1) node [left] {localstroge 登录记录};
      \draw [Circle-] (-1.75,-2.9) -- (-3,-2.9) node [left] {跳过登录,进入系统};
      \draw [Circle-] (-1.75,-3.3) -- (-3,-3.3) node [left] {进入注册界面};
    \end{scope}
  \end{tikzpicture}
  \caption{登录界面}
  \label{fig:登录界面}
\end{figure}

注册界面提供新用户注册功能,注册逻辑与常用手机 app 相同,必填数据选以 * 标注。后端获取注册数据后会进行如下判断: 用户是否已被注册,用户的权限组是否匹配。注册逻辑代码如下所示:

\begin{Java}
public RespBean registry(RegistryDTO registryDTO) {
    Long id = registryDTO.getId();
    String name = registryDTO.getName();
    String password = RSAUtils.decrypt(registryDTO.getPassword());
    String permission = registryDTO.getPermission();
    String email = registryDTO.getEmail();
    String avatarUrl = registryDTO.getAvatarUrl();
    System.out.println(password);
    // 数据库存在同名用户
    if (hasName(name))
    return RespBean.error("该用户名已被注册,请换一个");
    // 邮箱已被注册
    if (hasEmail(email))
    return RespBean.error("该邮箱已被注册");
    // 该 id 已经被注册
    if (userRepository.findById(id).isPresent())
        return RespBean.error("该编号对应的账户已注册");
    // 验证校园表用户权限
    Optional<UniversityUser> universityUser = universityUserRepository.findById(id);
    System.out.println(permission);
    if (!universityUser.isPresent())
        return RespBean.error("校园系统中不存在此学号/编号");
    else if (!universityUser.get().getIdentity().equals(permission))
        return RespBean.error("权限不匹配");
    // 注册用户
    Timestamp createTime = new Timestamp(System.currentTimeMillis());
    User newUser = User.builder().id(
            id).name(name)
            .password(password).permission(permission)
            .email(email).avatarUrl(avatarUrl).createTime(createTime).updateTime(createTime).build();
    System.out.println(newUser);
    userRepository.save(newUser);
    return RespBean.success("注册成功!");
}
\end{Java}



注册界面如图\ref{fig:注册界面}所示:

\begin{figure}[H]
  \small
  \centering
  \begin{tikzpicture}[font=\footnotesize]
    \begin{scope}[xshift=0cm]
      \node () at (-7,0) {};
      \node () at (7,0) {};
      \node [draw=black!60] (fig) at (0,0) {\includegraphics[width=4cm]{./figs/registry.jpg}};
      \draw [Circle-] (-1.75,-0.5) -- (-3,-0.5) node [left] {必填数据段,* 标识};
      \draw [Circle-] (-1.75,-2.5) -- (-3,-2.5) node [left] {可选数据段};
      \draw [Circle-] (-1.75,-3.3) -- (-3,-3.3) node [left] {注册按钮,提交表单};
    \end{scope}
  \end{tikzpicture}
  \caption{注册界面}
  \label{fig:注册界面}
\end{figure}

用户资料界面用于显示个人信息,同时提供一些快捷服务,用户可以在此修改个人信息。当用户点击``收藏''等按钮时,界面将弹出提示框,同时向后端发送数据请求,然后显示对应的数据。用户资料界面如图\ref{fig:用户资料界面}所示:

\begin{figure}[H]
  \small
  \centering
  \begin{tikzpicture}[font=\footnotesize]
    \begin{scope}[xshift=0cm]
      \node () at (-7,0) {};
      \node () at (7,0) {};
      \node [draw=black!60] (fig) at (0,0) {\includegraphics[width=4cm]{./figs/my.jpg}};
      \draw [Circle-] (1.75,3.6) -- (3,3.6) node [right] {切换主题,设置主题色};
      \draw [Circle-] (-1.75,2.88) -- (-3,2.88) node [left] {用户主要信息};
      \draw [Circle-] (-1.75,2.25) -- (-3,2.25) node [left] {修改签名};
      \draw [Circle-] (0.25,1.75) -- (3,1.75) node [right] {社区信息,点击可查看列表};
      \draw [Circle-] (1.75,1) -- (3,1) node [right] {活动面板,展示主要活动或提示};
      \draw [Circle-] (-1.75,0) -- (-3,0) node [left] {一些服务的快捷链接};
      \draw [Circle-] (-1.75,-2) -- (-3,-2) node [left] {设置项};
      \draw [Circle-] (-1.75,-4) -- (-3,-4) node [left] {其他模块链接};
    \end{scope}
  \end{tikzpicture}
  \caption{用户资料界面}
  \label{fig:用户资料界面}
\end{figure}

\subsubsection{数据推送}

数据推送模块包含首页界面。

用户进入首页可根据需求选择分区: ``为你推荐'', ``最新更新'', ``我的关注''。后端将根据用户选择采用对应算法传输可用数据到前端。同时用户也可手动搜索内容。显示主体内容的代码如下所示:

\begin{JavaScript}
const initModels = async () => {
  let promise;
  let data: ModelApiType[] = [];
  // 如果存在搜索内容,显示搜索内容
  if (searchContext) promise = searchModelsByTitleApi(searchContext);
  // 不存在搜索内容,显示分区内容
  else {
    // 获取推荐内容
    if (tab === HomeTab.recommend) promise = getRecommendModelApi();
    // 获取最新更新内容
    if (tab === HomeTab.new) promise = getLatestModelApi();
    // 获取用户订阅的内容
    if (tab === HomeTab.star) promise = getSubscribeModelApi(userId);
  }
  // 等待后端数据传输
  if (promise) data = (await (await promise).json()).data;
  const modelsArray = await modelAdapter(data);
  // 设置内容
  setModels(modelsArray);
};
\end{JavaScript}

主页界面中将显示各个模型的简要信息,包括预览图片,名称,作者等,用户点击对应卡片即可预览 3D 模型。``主页''界面如图\ref{fig:主页界面}所示:

\begin{figure}[H]
  \small
  \centering
  \begin{tikzpicture}[font=\footnotesize]
    \begin{scope}[xshift=0cm]
      \node () at (-7,0) {};
      \node () at (7,0) {};
      \node [draw=black!60] (fig) at (0,0) {\includegraphics[width=4cm]{./figs/home.jpg}};
      \draw [Circle-] (1.1,3.5) |- (3,3.8) node [right] {系统的一些更新提示};
      \draw [Circle-] (1.75,3.5) -- (3,3.5) node [right] {预留的邮箱功能,可接入邮箱系统};
      \draw [Circle-] (-1.75,3.5) -- (-3,3.5) node [left] {接入高德天气服务};
      \draw [Circle-] (-1.75,2.95) -- (-3,2.95) node [left] {搜索框};
      \draw [Circle-] (-1.75,2.5) -- (-3,2.5) node [left] {分区类型};
      \draw [Circle-] (-0.5,1) -- (-3,1) node [left] {模型卡,点击预览};
      \draw [Circle-] (0.75,1.75) -- (3,1.75) node [right] {模型封面};
      \draw [Circle-] (1,0.5) -- (3,0.5) node [right] {模型名称};
      \draw [Circle-] (-1.75,0.2) -- (-3,0.2) node [left] {作者};
      \draw [Circle-] (1.5,0.2) -- (3,0.2) node [right] {点赞数据};
      \draw [Circle-] (-1.75,-4) -- (-3,-4) node [left] {其他模块链接};
    \end{scope}
  \end{tikzpicture}
  \caption{主页界面}
  \label{fig:主页界面}
\end{figure}

\subsubsection{数据上传}

数据上传包含``上传''界面。

用户在下方点击加号按钮后点击模型上传即可进入``上传''界面,仅有管理员用户可进入该界面,其他用户尝试进入则会发出警告。

模型上传需填写必要数据,然后点击最下方的``预览按钮'',系统会渲染通过用户在表单中填入的数据渲染模型。用户根据预览效果选择上传模型或继续修改。

后端在接受到数据后会自行判断是否存在控制数据,如果存在则在模型控制表中添加内容,并将其关联到模型上。后端创建模型的代码如下:

\begin{Java}
public RespBean uploadModel(ModelUploadDTO dto) {
    Long modelControlId = null;
    // 存在控制信息
    if (dto.hasControl()) {
        ModelControl newModelControl = ModelControl.builder().autoRotateSpeed(dto.getAutoRotateSpeed())
                .background(dto.getBackground()).preset(dto.getPreset()).blur(dto.getBlur())
                .speed(dto.getSpeed()).rotationIntensity(dto.getRotationIntensity())
                .floatIntensity(dto.getFloatIntensity()).build();
        // 存入控制信息,同时获得 id
        newModelControl = modelControlRepository.save(newModelControl);
        modelControlId = newModelControl.getId();
    }
    // 创建模型,将控制信息 id 存入模型数据
    Model newModel = Model.builder().modelUrl(dto.getModelUrl()).coverUrl(dto.getCoverUrl())
            .title(dto.getTitle())
            .abstractInfo(dto.getInfo()).authorId(dto.getAuthorId())
            .controlId(modelControlId).build();
    modelRepository.save(newModel);
    return RespBean.success("成功上传模型");
}
\end{Java}

上传界面如图\ref{fig:上传界面}所示:

\begin{figure}[H]
  \small
  \centering
  \begin{tikzpicture}[font=\footnotesize]
    \node () at (-7,0) {};
    \node () at (7,0) {};
    \begin{scope}[xshift=-2.25cm]
      \node [draw=black!60] (fig) at (0,0) {\includegraphics[width=4cm]{./figs/upload.jpg}};
      \draw [Circle-] (1.5,3.75) -- (-3,3.75) node [left] {操作提示};
      \draw [Circle-] (-1.75,2.5) -- (-3,2.5) node [left] {必要上传数据};
      \draw [Circle-] (-1.5,-1) -- (-3,-1) node [left] {可选上传数据};
      \draw [Circle-] (-1.75,-4) -- (-3,-4) node [left] {其他模块链接};
    \end{scope}
    \begin{scope}[xshift=2.25cm]
      \node () at (-7,0) {};
      \node () at (7,0) {};
      \node [draw=black!60] (fig) at (0,0) {\includegraphics[width=4cm]{./figs/upload-2.jpg}};
      \draw [Circle-] (1.5,-2.65) -- (3,-2.65) node [right] {确认或返回};
      \draw [Circle-] (0,1.6) -- (3,1.6) node [right] {模型预览效果};
    \end{scope}
  \end{tikzpicture}
  \caption{上传界面}
  \label{fig:上传界面}
\end{figure}


\subsection{核心功能实现}
\subsubsection{WebGL 模块}

WebGL 模块包含模型预览界面。

用户进入模型预览界面后,系统将搭建场景,显示模型信息,请求模型信息数据:
\begin{itemize}
  \item 搭建场景: 系统通过模型 id 查模型表和模型控制表,获取模型 url 和相关配置信息。此时系统将显示场景。
  \item 请求模型: 系统获取模型 url 后异步请求模型数据,继而在界面上显示。
  \item 模型信息: 系统向后端请求模型的文本,点赞,作者等信息。
\end{itemize}

其中搭建场景,显示模型信息的功能同步进行,模型数据异步请求。前端请求数据,前端请求模型及其相关数据的代码如下:

\begin{JavaScript}
const initModel = async (id: string) => {
  // 获取模型基本数据
  let modelApiDate: ModelApiType | null = null;
  await (await getModelByIdApi(id)).json().then((response) => {
    modelApiDate = response.data;
  });
  // 查询模型被喜欢的数量
  let likeCount = 0;
  await (await getModelLikeCountsApi(String(modelApiDate!.id)))
    .json()
    .then((response) => (likeCount = response.data));
  // 将后端数据转换为前端数据
  const modelData: ThreeModelFields = {
    id: String(modelApiDate!.id),
    ......
  };
  // 获取模型作者信息
  const author = (await getUserById(String(modelApiDate!.authorId))) as User;
  const userData: UserShortFields = {
    id: author.id,
    ......
  };
  // 获取模型控制信息
  let modelControlApiDate: ModelControlApiType = {
    id: 0,
    ......
  };
  await (await getModelByIdApi(id)).json().then((response) => {
    modelControlApiDate = response.data;
  });
  // 将后端控制信息转换为前端数据
  const controlDate = modelControlApiDate as CommonControlFields &
    EnvironmentFields & FloatFields;
  // 创建模型实例
  const model: ThreeModel = ThreeModel.fromJson(modelData,userData,controlDate);
  // 设置模型信息
  setModel(model);
};
\end{JavaScript}


预览界面将模型信息分为简要信息与详细信息,简要信息仅以浮动卡片形式显现,用户可隐藏相关信息。详细信息则无法隐藏。用户在移动设备上可通过触摸移动位置,双指缩放模型大小,也可以通过控制板调整环境信息,例如背景贴图,背景模糊程度。模型预览界面如图\ref{fig:模型预览界面}所示:

\begin{figure}[H]
  \small
  \centering
  \begin{tikzpicture}[font=\footnotesize]
    \node () at (-7,0) {};
    \node () at (7,0) {};
    \begin{scope}[xshift=-2.25cm]
      \node [draw=black!60] (fig) at (0,0) {\includegraphics[width=4cm]{./figs/preview.jpg}};
      \draw [Circle-] (-0.5,3.75) -- (-3,3.75) node [left] {控制板};
      \draw [Circle-] (-1.75,3.5) -- (-3,3.5) node [left] {扩展功能按钮};
      \draw [Circle-] (-1.75,2.5) -- (-3,2.5) node [left] {场景背景};
      \draw [Circle-] (-1.5,0.2) -- (-3,0.2) node [left] {操作说明};
      \draw [Circle-] (-1.5,-1) -- (-3,-1) node [left] {快捷功能按钮};
      \draw [Circle-] (-1.75,-1.8) -- (-3,-1.8) node [left] {模型名称};
      \draw [Circle-] (-1.75,-2.2) -- (-3,-2.2) node [left] {模型简介};
      \draw [Circle-] (-1,-2.5) -- (-3,-2.5) node [left] {简要信息卡};
      \draw [Circle-] (-1.75,-3.1) -- (-3,-3.1) node [left] {模型点赞数据};
      \draw [Circle-] (0.75,-3.1) |- (-3,-2.8) node [left] {用户收藏按钮};
      \draw [Circle-] (1.5,-3.1) |- (-3,-3.4) node [left] {切换到详细信息};
      \draw [Circle-] (-1.75,-4) -- (-3,-4) node [left] {其他模块链接};
    \end{scope}
    \begin{scope}[xshift=2.25cm]
      \node () at (-7,0) {};
      \node () at (7,0) {};
      \node [draw=black!60] (fig) at (0,0) {\includegraphics[width=4cm]{./figs/preview-2.jpg}};
      \draw [Circle-] (1.75,3.3) -- (3,3.3) node [right] {切换环境背景};
      \draw [Circle-] (1.75,2.9) -- (3,2.9) node [right] {切换是否显示背景};
      \draw [Circle-] (1.75,2.5) -- (3,2.5) node [right] {设置背景模糊程度};
      \draw [Circle-] (0,0.5) -- (3,0.5) node [right] {模型本体};
      \draw [Circle-] (2,-0.5) -- (3,-0.5) node [right] {作者信息};
      \draw [Circle-] (1.5,-1) -- (3,-1) node [right] {模型发布与更新信息};
      \draw [Circle-] (1.75,-1.8) -- (3,-1.8) node [right] {模型详细信息};
      \draw [Circle-] (1.75,-4) -- (3,-4) node [right] {其他模块链接};
      \draw [Circle-] (-1.4,2.2) -- (3,2.2) node [right] {开关非必要组件};
      \draw [Circle-] (-1.4,1.6) -- (3,1.6) node [right] {开关操作说明};
      \draw [Circle-] (-1.4,1) -- (3,1) node [right] {在 AR 模式中显示};
    \end{scope}
  \end{tikzpicture}
  \caption{模型预览界面}
  \label{fig:模型预览界面}
\end{figure}

\subsubsection{AR 模块}

AR 模块包括 AR 预览与 AR 识别界面。

进入 AR 模式将提醒用户需要的设备条件,如果用户的设备不满足任意条件,AR 按钮将显示 AR unsupported。设备需要满足的条件如下:
\begin{itemize}
  \item 摄像头: 浏览器需要获取摄像头权限, 确保访问的是 https 协议网站,或者手动调整浏览器信任对应网站。否则浏览器无法获得摄像头权限。
  \item AR 服务: 安卓智能机需开启 AR 服务,请在 google play 安装 “面向 AR 的Google Play 服务” 插件。苹果设备开启 ARKit 服务。
  \item 浏览器: 尽量使用 chrome 浏览器,AR 功能至少需要 chrome 80+ 版本。安卓设备尽量保持在系统 11+ 版本。
  \item 网络环境: 出于优化考虑,系统不会一次推送所有数据,AR 部分功能需要实时获取网络资源。
\end{itemize}

AR 提示界面如图\ref{fig:AR提示界面}所示:

\begin{figure}[H]
  \small
  \centering
  \begin{tikzpicture}[font=\footnotesize]
    \begin{scope}[xshift=0cm]
      \node () at (-7,0) {};
      \node () at (7,0) {};
      \node [draw=black!60] (fig) at (0,0) {\includegraphics[width=4cm]{./figs/ar.jpg}};
      \draw [Circle-] (-0.5,1) -- (-3,1) node [left] {AR 提示信息};
      \draw [Circle-] (-1.5,-1.5) -- (-3,-1.5) node [left] {快捷操作按钮};
      \draw [Circle-] (-0.5,-3.25) -- (-3,-3.25) node [left] {AR 按钮};
      \draw [Circle-] (-1.75,-4) -- (-3,-4) node [left] {其他模块链接};
    \end{scope}
  \end{tikzpicture}
  \caption{AR提示界面}
  \label{fig:AR提示界面}
\end{figure}

进入 AR 预览界面后,系统会请求手机摄像头权限,用户确认之后,系统将实时获取显示影像,并将 3D 模式显示在界面中,用户可以通过控制板或手动调整操作模型。系统将实时计算模型所在位置,AR 预览界面如图\ref{fig:AR预览界面}所示:

\begin{figure}[H]
  \small
  \centering
  \begin{tikzpicture}[font=\footnotesize]
    \node () at (-7,0) {};
    \node () at (7,0) {};
    \begin{scope}[xshift=-2.25cm]
      \node [draw=black!60] (fig) at (0,0) {\includegraphics[width=4cm]{./figs/ar-preview.jpg}};
      \draw [Circle-] (-0.5,4) -- (-3,4) node [left] {控制板};
      \draw [Circle-] (-1.75,3.75) -- (-3,3.75) node [left] {扩展功能按钮};
      \draw [Circle-] (-1.75,3.1) -- (-3,3.1) node [left] {开关非必要组件};
      \draw [Circle-] (-1.75,2.5) -- (-3,2.5) node [left] {允许/禁止触控};
      \draw [Circle-] (-1.75,1.9) -- (-3,1.9) node [left] {开关操作说明};
      \draw [Circle-] (-1.5,0) -- (-3,0) node [left] {模型本体};
      \draw [Circle-] (0.5,-0.5) -- (-3,-0.5) node [left] {触摸操作轴};
      \draw [Circle-] (-1.5,-2) -- (-3,-2) node [left] {现实场景};
      \draw [Circle-] (-0.25,-3.25) -- (-3,-3.25) node [left] {AR 按钮};
      \draw [Circle-] (-1.75,-4) -- (-3,-4) node [left] {其他模块链接};
    \end{scope}
    \begin{scope}[xshift=2.25cm]
      \node [draw=black!60] (fig) at (0,0) {\includegraphics[width=4cm]{./figs/ar-preview-2.jpg}};
      \draw [Circle-] (1.75,3) -- (3,3) node [right] {设置模型位置参数};
      \draw [Circle-] (1.75,2) -- (3,2) node [right] {设置模型旋转参数};
      \draw [Circle-] (1.75,1.3) -- (3,1.3) node [right] {设置模型尺寸};
    \end{scope}
  \end{tikzpicture}
  \caption{AR预览界面}
  \label{fig:AR预览界面}
\end{figure}

进入 AR 识别界面后,系统首先会加载常规标记数据,同时尝试访问设备定位权限,如果成功获取定位信息,加载地理区域内的特定标记数据。

AR 识别功能支持的两种识别技术优缺点如下:
\begin{itemize}
  \item 标记捕捉: 图形具有特定要求,数据量小,适合较普遍的现实图像。
  \item 图像识别: 数据量大,可识别任何特想,适合特殊的现实图像。
\end{itemize}

系统使用标记捕捉技术识别常规图像,进入系统后则加载所有的标记捕捉数据。系统主要使用图像识别技术识别特定场景,例如图书馆,名人雕像等。由于图像识别需要大量数据,因此系统仅加载区域内数据,具体实现上根据地理坐标加载。

前端尝试获取定位信息,加载常规标记数据和地理区域内特定标记数据的代码如下所示:

\begin{JavaScript}
const initMarkerData = () => {
  // 成功获取定位的回调函数: 向后端请求常规数据和区域内特定数据
  const successCallBack = async (data: any) => {
    const lon = data.coords.longitude;
    const lat = data.coords.latitude;
    const markers: Marker[] = [];
    // 请求特定数据
    await (await getMarkersByGeoApi(lon, lat))
    .json()
    .then(async (response) => {
      const markersApiData: MarkerApiType[] = response.data;
      for (const markerApiData of markersApiData) {
        const marker = await markerAdapter(markerApiData);
        markers.push(marker);
        }
        });
    // 请求常规数据
    await (await getCommonMarkersApi()).json().then(async (response) => {
      const markersApiData: MarkerApiType[] = response.data;
      for (const markerApiData of markersApiData) {
        const marker = await markerAdapter(markerApiData);
        markers.push(marker);
      }
    });
    setMarkers(markers);
    setInit(true);
  };
  // 获取定位失败的回调函数: 仅仅向后端请求常规数据
  const errorCallBack = async () => {
    // 代码和成功回调函数的请求常规数据类似
    ......
  };
  // 获取定位数据
  navigator.geolocation.getCurrentPosition(successCallBack, errorCallBack);
};
\end{JavaScript}

针对地理区域内特定标记数据,后端会在数据库中根据经纬度坐标进行判断,默认加载偏差在 2km 内的特定数据(即对应经纬度 0.02 内的数据),后端代码如下:

\begin{JavaScript}
@Override
public RespBean getMarkersByGeo(Float lon, Float lat) {
    // 范围: 0.01 经纬度,即 1km
    Float len = 0.01f;
    List<Marker> markersList = markerRepository.findAllByLonBetweenAndLatBetween(
        lon - len, lon + len, lat - len, lat + len);
    return RespBean.success("获取识别标记数据", markersList);
}
\end{JavaScript}

加载完所有标记数据后,系统实时匹配摄像机传输的图像,如果匹配上对应的标记,界面将弹出简略信息卡,用户点击即可查看详细信息。如果存在模型会在标记法线上方显示模型。AR 识别界面如图\ref{fig:AR识别界面}所示:

\begin{figure}[H]
  \small
  \centering
  \begin{tikzpicture}[font=\footnotesize]
    \node () at (-7,0) {};
    \node () at (7,0) {};
    \begin{scope}[xshift=-2.25cm]
      \node [draw=black!60] (fig) at (0,0) {\includegraphics[width=4cm]{./figs/ar-identify.jpg}};
      \draw [Circle-] (-1.75,3.75) -- (-3,3.75) node [left] {扩展功能按钮};
      \draw [Circle-] (-1.75,3.1) -- (-3,3.1) node [left] {开关非必要组件};
      \draw [Circle-] (-1.75,2.5) -- (-3,2.5) node [left] {开关操作说明};
      \draw [Circle-] (0,0) -- (-3,0) node [left] {显示的模型};
      \draw [Circle-] (0.5,-0.5) -- (-3,-0.5) node [left] {匹配到的标记};
      \draw [Circle-] (-1.7,-2.3) -- (-3,-2.3) node [left] {标记简略信息卡};
      \draw [Circle-] (-0.25,-3.25) -- (-3,-3.25) node [left] {AR 按钮};
      \draw [Circle-] (-1.75,-4) -- (-3,-4) node [left] {其他模块链接};
    \end{scope}
    \begin{scope}[xshift=2.25cm]
      \node [draw=black!60] (fig) at (0,0) {\includegraphics[width=4cm]{./figs/ar-identify-2.jpg}};
      \draw [Circle-] (0.5,4) -- (3,4) node [right] {控制板};
      \draw [Circle-] (1.75,3.6) -- (3,3.6) node [right] {设置模型背景};
      \draw [Circle-] (1.75,0) -- (3,0) node [right] {标记详细信息};
    \end{scope}
  \end{tikzpicture}
  \caption{AR识别界面}
  \label{fig:AR识别界面}
\end{figure}