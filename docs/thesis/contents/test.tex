\section{系统测试}

软件测试,测是验证软件功能、性能;试是验证软件是否有非功能性的异常。软件测试是软件开发必不可少的过程,目的是不断优化软件系统,让软件更加的完善。

\subsection{测试环境}

硬件环境:
\begin{itemize}
  \item 服务器: 本机服务器(4核8线程,24GB,10M贷款,100G 硬盘);
  \item 移动设备: 安卓手机(Snapdragon 885 处理器, Android 11 操作系统);
\end{itemize}

软件环境:
\begin{itemize}
  \item 数据库: MySQL 8.0;
  \item Java: JDK17;
  \item 浏览器: Chroe112(移动版)。
\end{itemize}

\subsection{功能测试}

\subsubsection{用户认证模块测试}

用户认证模块的功能包括用户登录,用户注册,用户信息增改。测试用例和结果如表\ref{table:用户认证模块测试用例表}所示:

\begin{table}[H]
  \centering
  \small
  \renewcommand\arraystretch{1.1}
  \caption{用户认证模块测试用例表}
  \label{table:用户认证模块测试用例表}
  \setlength{\tabcolsep}{4mm}
  \begin{tabular}{|p{4.5cm}|p{6.5cm}|p{1.5cm}|}
    \hline \textbf{功能描述} & \multicolumn{2}{l|}{用户登录,注册,信息修改} \\
    \hline \textbf{用例目的} & \multicolumn{2}{l|}{测试模块功能的可用性} \\
    \hline \textbf{前提条件} & \multicolumn{2}{l|}{用户时校园教职工或学生,在校园数据表中有记录} \\
    \hline \textbf{输出/动作} & \textbf{期望的输出/响应} & \textbf{实际情况} \\
    \hline 进入登录界面,在表单中输入用户名或邮箱以及密码。点击登录按钮。 & 前端对输入数据进行判断,根据字符关键字判断是用户名还是密码。对用户名/邮箱和密码进行格式校验。若失败,前端给出提示。若成功将数据发送到后端验证,若后端验证失败,则前端界面显示对应信息,若成功,则跳转至系统主页面。 & 符合  \\
    \hline 进入注册界面,在表单中输入工号/学号,密码等必填信息;选择性填入可填信息。点击注册按钮。 & 前端对输入数据进行格式判断。若失败,前端给出提示。若成功将数据发送到后端验证,后端查询学校用户表,根据 id 匹配注册信息,若后端验证失败,则前端界面显示对应信息,若成功,则跳转至登录页面。 & 符合  \\
    \hline 登录账户后,在``我的''界面修改个人信息,点击提交按钮。 & 前端对输入数据进行格式判断。若失败,前端给出提示。若成功将数据发送到后端验证,后端将数据存储到对应表格,若失败,前端给出错误原因,若成功,更新用户信息。 & 符合  \\
    \hline 在登录时,勾选 ``记住我'' 选项。 & 退出系统后,在一周内登录系统无需手动登录,系统自动跳转至 ``我的'' 界面。 & 符合  \\
    \hline
  \end{tabular}
\end{table}

\subsubsection{数据推送模块测试}

数据推送模块的功能包括模型推送与模型查询。测试用例和结果如表\ref{table:数据推送模块测试用例表}所示:

\begin{table}[H]
  \centering
  \small
  \renewcommand\arraystretch{1.1}
  \caption{数据推送模块测试用例表}
  \label{table:数据推送模块测试用例表}
  \setlength{\tabcolsep}{4mm}
  \begin{tabular}{|p{4.5cm}|p{6.5cm}|p{1.5cm}|}
    \hline \textbf{功能描述} & \multicolumn{2}{l|}{模型推送,模型查询} \\
    \hline \textbf{用例目的} & \multicolumn{2}{l|}{测试模块功能的可用性} \\
    \hline \textbf{前提条件} & \multicolumn{2}{l|}{用户进入系统} \\
    \hline \textbf{输出/动作} & \textbf{期望的输出/响应} & \textbf{实际情况} \\
    \hline 进入主页界面,选择分区: “为你推荐”, “最新更新”, “我的关注”。 & 前端将对应类型传输到后端,后端根据类型选择不同算法,分别采用后端三种不同的算法: 点赞数较高模型,最新更新模型,用户关注作者模型。返回数据在前端显示。 & 符合  \\
    \hline 进入主页界面,在搜索框输入模型名称,点击搜索图标。 & 前端对搜索格式进行检测,若不符合要求则给出提示信息。若符合则后端根据搜索内容查模型表模型名称,然后返回查询到的结果数据到前端能显示。 & 符合  \\
    \hline
  \end{tabular}
\end{table}

\subsubsection{数据上传模块测试}

数据上传模块的功能包括模型上传与AR标记上传。测试用例和结果如表\ref{table:数据上传模块测试用例表}所示:

\begin{table}[H]
  \centering
  \small
  \renewcommand\arraystretch{1.1}
  \caption{数据上传模块测试用例表}
  \label{table:数据上传模块测试用例表}
  \setlength{\tabcolsep}{4mm}
  \begin{tabular}{|p{4.5cm}|p{6.5cm}|p{1.5cm}|}
    \hline \textbf{功能描述} & \multicolumn{2}{l|}{模型上传,AR标记上传} \\
    \hline \textbf{用例目的} & \multicolumn{2}{l|}{测试模块功能的可用性} \\
    \hline \textbf{前提条件} & \multicolumn{2}{l|}{用户进入系统} \\
    \hline \textbf{输出/动作} & \textbf{期望的输出/响应} & \textbf{实际情况} \\
    \hline 进入上传界面,选择模型上传。在表格中填入模型 URL 等必要信息;选择性填入可选信息。点击上传按钮 & 前端对输入数据进行格式判断,对模型 URL 有效性检查。若检查失败,前端给出提示。若成功将数据发送到后端存储,后端检查是否存在相同数据,若存在返回错误信息,若存储成功,前端给出成功提示。 & 符合  \\
    \hline 进入上传界面,选择AR标记上传。在表格中填入标记 URL,标记类型等必要信息;选择性填入可选信息。点击上传按钮 & 前端对输入数据进行格式判断,对标记 URL 有效性检查。若检查失败,前端给出提示。若成功将数据发送到后端存储,后端检查是否存在相同数据,若存在返回错误信息,若成功,分别将数据存储到标记表,标记信息表。存储完成后前端给出成功提示。 & 符合  \\
    \hline
  \end{tabular}
\end{table}

\subsubsection{WebGL 模块测试}

WebGL 模块的功能包括模型显示,模型控制功能。测试用例和结果如表\ref{table:WebGL 模块测试用例表}所示:

\begin{table}[H]
  \centering
  \small
  \renewcommand\arraystretch{1.1}
  \caption{WebGL 模块测试用例表}
  \label{table:WebGL 模块测试用例表}
  \setlength{\tabcolsep}{4mm}
  \begin{tabular}{|p{4.5cm}|p{6.5cm}|p{1.5cm}|}
    \hline \textbf{功能描述} & \multicolumn{2}{l|}{模型显示,模型控制} \\
    \hline \textbf{用例目的} & \multicolumn{2}{l|}{测试模块功能的可用性} \\
    \hline \textbf{前提条件} & \multicolumn{2}{l|}{用户进入系统} \\
    \hline \textbf{输出/动作} & \textbf{期望的输出/响应} & \textbf{实际情况} \\
    \hline 进入模型预览界面,查看某个模型。 & 后端将模型信息,控制信息,环境信息等传输到前端。前端异步加载环境,模型,模型信息。用户可通过按钮切换模型摘要或模型详情模式,也可隐藏非必要卡片。 & 符合  \\
    \hline 进入模型预览界面,通过触摸操作模型,通过控制板调整数据操作模型。 & 根据用户操作,模型做出移动,缩放等响应。通过控制板的操作,场景将更换对应配置或显示新的场景。 & 符合  \\
    \hline
  \end{tabular}
\end{table}

\subsubsection{AR 模块测试}

AR 模块的功能包括AR模型预览,AR识别功能。其中 AR 识别分为图像识别与标记捕捉。测试用例和结果如表\ref{table:AR模块测试用例表}所示:

\begin{table}[H]
  \centering
  \small
  \renewcommand\arraystretch{1.1}
  \caption{AR模块测试用例表}
  \label{table:AR模块测试用例表}
  \setlength{\tabcolsep}{4mm}
  \begin{tabular}{|p{4.5cm}|p{6.5cm}|p{1.5cm}|}
    \hline \textbf{功能描述} & \multicolumn{2}{l|}{AR预览,AR识别} \\
    \hline \textbf{用例目的} & \multicolumn{2}{l|}{测试模块功能的可用性} \\
    \hline \textbf{前提条件} & \multicolumn{2}{l|}{用户进入系统} \\
    \hline \textbf{输出/动作} & \textbf{期望的输出/响应} & \textbf{实际情况} \\
    \hline 进入AR预览界面,通过摄像头在虚拟现实场景中显示模型。 & 系统获取摄像头权限并显示现实场景。将虚拟物体模型投射到真实场景中提供预览功能。 & 符合  \\
    \hline 进入AR预览界面,通过操纵轴触摸操作模型位置,控制模型进行旋转。 & 模型根据用户触摸操作做出响应,对应位置或旋转信息改变并显示在界面上。 & 符合  \\
    \hline 进入AR预览界面,通过控制板改变模型或场景信息。 & 模型或场景根据改变的信息实时应用,并显示在界面上。 & 符合  \\
    \hline 进入AR识别界面,摄像头对准可识别的图像,等待系统响应。 & 系统匹配到图像信息后,如果存在预定义的 3D 模型,则显示在界面上;如果存在标记信息,则界面弹出简要标记卡片。 & 符合  \\
    \hline 进入AR识别界面,摄像头对准可识别的标记,等待系统响应。 & 系统匹配到标记信息后,如果存在预定义的 3D 模型,则显示在界面上;如果存在标记信息,则界面弹出简要标记卡片。 & 符合  \\
    \hline 弹出简要标记卡片后点击查看详细信息。 & 系统弹出详细信息卡,同时停止图像识别功能,在关系信息卡后重新进行图像识别。 & 符合  \\
    \hline
  \end{tabular}
\end{table}

\subsection{性能测试}

在性能测试过程中,针对需求分析中的非功能需求,对系统进行性能测试。具体测试方法为采用测试设备对系统各个功能接口同时进行测试,单个接口的测试量为 20 次,同类接口总体测试量大于 100,获得测试结果后取平均响应时间。测试用例和结果如表\ref{table:性能测试表}所示。

\begin{table}[H]
  \centering
  \small
  \renewcommand\arraystretch{1.1}
  \caption{性能测试表}
  \label{table:性能测试表}
  \setlength{\tabcolsep}{4mm}
  \begin{tabular}{|p{4.5cm}|p{4cm}|p{4cm}|}
    \hline \textbf{输入动作} & \textbf{期望的输出/响应} & \textbf{实际情况} \\
    \hline 切换系统界面 & 平均响应时间 <1s & 平均响应时间 <0.5s  \\
    \hline 点击按钮,弹出弹框并显示内容 & 平均响应时间 <1s & 平均响应时间 <0.5s  \\
    \hline 主页选择不同数据分析,等待返回内容 & 平均响应时间 <2s & 平均响应时间 <1.5s  \\
    \hline 主页搜索模型,等待返回内容 & 平均响应时间 <2s & 平均响应时间 <2s  \\
    \hline 模型预览界面加载场景,并开始异步加载模型 & 平均响应时间 <2s & 平均响应时间 <1s  \\
    \hline AR 预览界面开启摄像头,开始异步加载模型 & 平均响应时间 <2s & 平均响应时间 <1s  \\
    \hline AR 识别界面开启识别图像信息 & 平均响应时间 <2s & 平均响应时间 <1.5s  \\
    \hline AR 识别界面开启识别标记信息 & 平均响应时间 <1.5s & 平均响应时间 <1s  \\
    \hline
  \end{tabular}
\end{table}