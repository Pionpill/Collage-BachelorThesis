\section{总结}

本文以 WebGL 与 AR 技术为核心,介绍了 AR 交互系统的研究现状与研究意义,探索 AR 系统在 Web 领域的可行性,在完成技术选择,需求分析,系统整体设计,模块详细设计后,实现了基于 WebGL 的校园交互系统,并通过系统测试验证了系统的可用性,稳定性。系统以移动端浏览器为载体,调用操作系统底层 AR 服务,通过现有的 WebGL 与 AR 技术实现了模型预览,AR 预览,AR 识别等核心功能。

本系统的主要成果包括:
\begin{itemize}
  \item 基于 ar.js 提供的标识识别与图像识别算法,开发出了 AR 识别系统,能够对校园特定场景与物体进行识别并提供对应信息。
  \item 基于 three.js 技术,开发了 Web 端三维模型显示系统,并提供了一定的交互功能。结合 ar.js 基础开发了 AR 交互功能。给用户提供了一种更生动,更直观的交互方式。
  \item 基于 React,MaterialUI 技术,搭建了有好的前端系统,以此为依托开发了核心的 WebGL 与 AR 功能。用户无需下载任何应用即可访问系统。
  \item 基于 SpringBoot 技术搭建了后端系统,向前端提供稳定数据。
\end{itemize}

本文研究的 AR 系统可以提供更完善的预览服务,基于用户更丰富的视觉体验,本系统可拓展到教育,建筑,工业等各个领域,并于已有系统结合,有效提高工作,学习效率。本系统完全基于开源框架,且基于 Web 系统。决绝传统 AR 服务闭源,兼容性差等特点。

本系统同样存在部分问题,系统依赖于高性能智能设备,需要设备提供 AR 服务,需要用户具备一定的 3D 模型知识。